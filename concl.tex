\section{Conclusion}

In this work, we aim to mitigate the memory pressure in current service-oriented data processing systems. We analyze the memory usage of various function APIs enclosed in these systems, and build three coarse-grained models to classify the function APIs. Further, we propose to use the memory usage rate as a uniform criteria to measure the impact of a task on memory pressure in the service-oriented systems. Based on the memory usage rate, we develop a scheduler called MURS. MURS can suspend the tasks that cause heave memory pressure, and speed up the tasks with light memory pressure. MURS can be implemented in most data processing systems running in the service-oriented context. We conduct the extensive experiments to evaluate the effectiveness of the proposed methods and MURS. The results show that comparing with Spark, MURS can reduce the execution time of tasks by up to 65.8\%, mitigate the memory pressure by up to 81\% and avoid the task spill by approximately 90\%.